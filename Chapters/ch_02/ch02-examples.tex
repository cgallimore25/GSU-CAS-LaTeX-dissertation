\documentclass[../../main.tex]{subfiles}  % Same level as chapter file, so same path
\graphicspath{{\subfix{../Plots/}}}

\begin{document}

\section{Section structure}

	\subsection{subsection}

	\subsubsection{subsubsection}

\section{Content examples}

	\subsection{Citations}

	Woo! 
	Chapter 2!
	Here's how to cite a random paper\supercite{aartsSolutionDependencyUsing2014}. 
	According to \citeauthor{aartsSolutionDependencyUsing2014} \citeyear{aartsSolutionDependencyUsing2014}, you can also do it this way.
	See bibliography.bib for doing references.

	\subsection{Abbreviations}

	The \verb|acronym| package helps you manage the insertion of abbreviations, applied programmatically when inserted with the commands below. 
	Take a look at the \verb|abbreviations.tex| file to see how they're defined.
	Review the documentation for exhaustive command list.

	\begin{verbatim}
		\ac{ERP}		 % first instance expanded
		\ac{ERP}		 % second + thereafter truncated
		\acp{ERP}		% plural
		\Acf{ERP}		% force full expansion, capitalize
	  \end{verbatim}

	\ac{ERP}, \ac{ERP}, \acp{ERP}, \Acf{ERP}	

	You can even compose acronyms from others, for example, \ac{NMDA} elicits responses from \acp{NMDAR}. 
	A variety of neuropsychiatric disease states are associated with \ac{NRH}.

	\subsection{SI units}

	The \verb|siunitx| package allows us to insert units in a typographically consistent way, ensuring proper height, spacing, and proper formatting of units (e.g. prevention of linebreaks).
	It's desirable in this context for much the same reasons as the \verb|acronym| package.

	% millimolar is not an SI unit, but we can define it ourselves
	% defined in usercommands.tex as
	% \DeclareSIUnit{\mM}{mM}        % millimolar

	The concentration of drug was \SI[per-mode=fraction]{0.02}{\mol\per\L} (\SI{20}{\mM})).
	After a \SI{30}{\minute} period for pharmacological action, mice were placed in a chamber, anesthetized under \qty{3}{\percent} isoflurane, then fixed in a nose cone delivering the same anesthetic mixture. 
	Samples were extracted, tin foil wrapped, and placed in a \SI{-80}{\degreeCelsius} freezer for 25 minutes to achieve firmness. We also used 

	\subsection{Equations}

	Here I'll show you how to use some math, incorporating some more examples of units and abbreviations.

	\Acf{CSD}, $\Phi$, was calculated by a numerical approximation of the second spatial derivative of voltage, expressed as:

	\begin{equation}
		\Phi_{t, c} = -\frac{\phi_{t, c+1} - 2\phi_{t, c} + \phi_{t, c-1}}{\Delta c^2 \times g}
	\end{equation}

	where $\phi_{t, c}$ represents recorded voltages at time-point $t$ and channel $c$ on the linear electrode array, such that constant $\Delta c$ is the spacing between adjacent contacts. 
	The constant $g$ represents the conductivity of the extracellular medium, defined for these analyses as \SI[per-mode=fraction]{0.3}{\siemens\per\m}, consistent with putative bounds of extracellular impedance measurements (\qtyrange[range-units=single,range-phrase=-]{\sim 0.3}{0.6}{\siemens\per\m}) for frequencies below \SI{10}{\kHz}. 
	% notice how \siemens\per\m, when attached to a constant, is a fraction, but uses exponent notation when expressed for a range
	Checkout the code that generates these units. 

	\subsection{Figures}

	\begin{figure*}[hb] % place approximately here 'h', OR at the bottom 'b' of the page (adding 'b' relaxes the demand)
		\centering
		\includegraphics[height =4in]{./Plots/nature.jpg} 
		\caption{An individual figure!}
	\end{figure*}
			
	\begin{figure}[H] % place float exactly here 'H'
		\subfloat[\label{fig:HD8538_ellplot}]{\includegraphics[height =2.5in]{./Plots/nature.jpg}} 
		\subfloat[\label{fig:HD8538_phot}]{\includegraphics[height =2.5in]{./Plots/rocks.jpg}} \\
		\subfloat[\label{fig:HD8538_vis}]{\includegraphics[height =2.5in]{./Plots/rocks.jpg}}
		\subfloat[\label{fig:HD8538_HRD}]{\includegraphics[height =2.5in]{./Plots/nature.jpg}}
		\caption{Multiple figures!}
	\end{figure}


	\subsection{Tables}
	% The below was in the original latex template, which uses the `deluxetable.sty` package.
% The deluxetable package sucks, defining commands that collide with definitions in other packages.
% I avoid it at all costs, and provide an alternative for making tables with more friendly pkgs
% \begin{landscape}
%     \centering
%     \begin{longtable}{cccccccccccccc}
%         \caption{Insert Table Caption here}\\
%         \label{tab:disk}\\
%         \hline\endhead  % header material
%         \hline\endfoot  % footer material
%         \hline
%         Blah & Blah & Blah \\
%         \hline
%         Stuff & Things & etc. \\
%         ... & ... & ... \\
%     \end{longtable}
% \end{landscape}

% New table example -- see `usercommands.tex` for customizable table formatting specifications
\begin{landscape}
    \begin{table}[H]
        \centering
        \caption{Insert Table Caption here}
        \begin{tabular}{@{} l c c c @{}}
        \toprule
        \thead{Blah} & \thead{Blah} & \thead{Blah}\\
        \midrule
        Stuff & Things & etc.\\
        ...   & ...    & ... \\
        \bottomrule
        \end{tabular}
        
        \label{tab:my_table}
    \end{table}
\end{landscape}


\end{document}