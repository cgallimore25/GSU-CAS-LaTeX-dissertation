%%%%%%%%%%%%%%%%%%%% The backmatter goes in this file %%%%%%%%%%%%%%%%%%%%%

% The bibliography starts here.

% original CAS template used the below for 'natbib' -- uncomment the below two \bibliography{} commands and comment out \setlength and \printbibliography for natbib
% \bibliographystyle{apj}             % Please learn to use the formatting of Latex's Bibtex. It will make your life easier.
% \bibliography{bibliography}
% \bibliography{apj-jour,dissref}       % "paper.bib" contains all my references. "apj-jour.bib" contains abbreviations of journals.

% The biblatex bibliography starts here.
\setlength\bibitemsep{0pt}   % if using biblatex
\printbibliography[heading=bibintoc, title=\bibtocname]

\clearpage


\beforechapterheadname{Appendix}         % Optional text to put in front of the chapter number.
% \afterchapterheadname{}                % Optional text to put after the

\chapter*{APPENDICES}  % Create a proper chapter heading
\addcontentsline{toc}{chapter}{APPENDICES}  % Add to TOC at chapter level

\appendix

% This is crucial - directly modify how ToC entries will appear; need to do this BEFORE the sections are created
\addtocontents{toc}{%
  \protect\renewcommand{\protect\cftsecpresnum}{Appendix }%
  \protect\setlength{\protect\cftsecnumwidth}{7em}%
}


% \documentclass[../../main.tex]{subfiles}  % Two levels up to main.tex
\begin{document}

\setupappendix{A}  % See `usercommands.tex` for definitions

\section{Something}
This is the appendix!

\end{document}
\subfile{./Chapters/appendices/Appendix_A}
\clearpage
\subfile{./Chapters/appendices/Appendix_B}
\clearpage

% If you have only one appendix chapter, use the command
% \begin{appendix}...\end{appendix} instead.
% This takes care of the requirement (of the Graduate Office) for one
% appendix chapter to be labeled as 'Appendix', not 'Appendix A'.

%\begin{appendix}
%  \input{appendixI}                     % Your appendices go here.
%  \input{appendix}                     %named "appendix.tex"
%\end{appendix}
