%%%%%%%%%%%%%%%%%%%% The backmatter goes in this file %%%%%%%%%%%%%%%%%%%%%

% The bibliography starts here.

% original CAS template used the below for 'natbib' -- uncomment the below two \bibliography{} commands and comment out \setlength and \printbibliography for natbib
% \bibliographystyle{apj}             % Please learn to use the formatting of Latex's Bibtex. It will make your life easier.
% \bibliography{bibliography}
% \bibliography{apj-jour,dissref}       % "paper.bib" contains all my references. "apj-jour.bib" contains abbreviations of journals.

% The biblatex bibliography starts here.
\setlength\bibitemsep{0pt}   % if using biblatex
\printbibliography[heading=bibintoc, title=\bibtocname]

\clearpage


\beforechapterheadname{Appendix}         % Optional text to put in front of the chapter number.
% \afterchapterheadname{}                % Optional text to put after the

\chapter*{APPENDICES}  % Create a proper chapter heading
\addcontentsline{toc}{chapter}{APPENDICES}  % Add to TOC at chapter level

\appendix

% This is crucial - directly modify how ToC entries will appear; need to do this BEFORE the sections are created
\addtocontents{toc}{%
  \protect\renewcommand{\protect\cftsecpresnum}{Appendix }%
  \protect\setlength{\protect\cftsecnumwidth}{7em}%
}


% \documentclass[../../main.tex]{subfiles}  % Two levels up to main.tex
\begin{document}

\section{Something}
This is the appendix!

\end{document}
\subfile{./Chapters/appendices/Appendix_A}
\clearpage
\subfile{./Chapters/appendices/Appendix_B}
\clearpage

% If you have only one appendix chapter, use the command
% \begin{appendix}...\end{appendix} instead.
% This takes care of the requirement (of the Graduate Office) for one
% appendix chapter to be labeled as 'Appendix', not 'Appendix A'.

%\begin{appendix}
%  \input{appendixI}                     % Your appendices go here.
%  %%
%% This is file `appendix.sty',
%% generated with the docstrip utility.
%%
%% The original source files were:
%%
%% appendix.dtx  (with options: `usc')
%%
%% -----------------------------------------------------------------
%%   Author: Peter Wilson (CUA) now at peter.r.wilson@boeing.com until June 2004
%%                              (or at: pandgwilson at earthlink dot net)
%%   Copyright 1998 --- 2004 Peter R. Wilson
%%
%%   This work may be distributed and/or modified under the
%%   conditions of the LaTeX Project Public License, either
%%   version 1.3 of this license or (at your option) any
%%   later version.
%%   The latest version of the license is in
%%      http://www.latex-project.org/lppl.txt
%%   and version 1.3 or later is part of all distributions of
%%   LaTeX version 2003/06/01 or later.
%%
%%   This work has the LPPL maintenance status "author-maintained".
%%
%%   This work consists of the files listed in the README file.
%% -----------------------------------------------------------------
%%
\NeedsTeXFormat{LaTeX2e}
\ProvidesPackage{appendix}[2002/08/06 v1.2 extra appendix facilities]

\newif\if@chapter@pp\@chapter@ppfalse
\newif\if@knownclass@pp\@knownclass@ppfalse
\@ifundefined{chapter}{%
  \@ifundefined{section}{}{\@knownclass@pptrue}}{%
  \@chapter@pptrue\@knownclass@pptrue}
\providecommand{\phantomsection}{}
\newcounter{@pps}
  \renewcommand{\the@pps}{\alph{@pps}}
\newif\if@pphyper
  \@pphyperfalse
\AtBeginDocument{%
  \@ifpackageloaded{hyperref}{\@pphypertrue}{}}

\newif\if@dotoc@pp\@dotoc@ppfalse
\newif\if@dotitle@pp\@dotitle@ppfalse
\newif\if@dotitletoc@pp\@dotitletoc@ppfalse
\newif\if@dohead@pp\@dohead@ppfalse
\newif\if@dopage@pp\@dopage@ppfalse
\DeclareOption{toc}{\@dotoc@pptrue}
\DeclareOption{title}{\@dotitle@pptrue}
\DeclareOption{titletoc}{\@dotitletoc@pptrue}
\DeclareOption{header}{\@dohead@pptrue}
\DeclareOption{page}{\@dopage@pptrue}
\ProcessOptions\relax
\newcommand{\@ppendinput}{}
\if@knownclass@pp\else
  \PackageWarningNoLine{appendix}%
    {There is no \protect\chapter\space or \protect\section\space command.\MessageBreak
     The appendix package will not be used}
  \renewcommand{\@ppendinput}{\endinput}
\fi
\@ppendinput

\newcommand{\appendixtocon}{\@dotoc@pptrue}
\newcommand{\appendixtocoff}{\@dotoc@ppfalse}
\newcommand{\appendixpageon}{\@dopage@pptrue}
\newcommand{\appendixpageoff}{\@dopage@ppfalse}
\newcommand{\appendixtitleon}{\@dotitle@pptrue}
\newcommand{\appendixtitleoff}{\@dotitle@ppfalse}
\newcommand{\appendixtitletocon}{\@dotitletoc@pptrue}
\newcommand{\appendixtitletocoff}{\@dotitletoc@ppfalse}
\newcommand{\appendixheaderon}{\@dohead@pptrue}
\newcommand{\appendixheaderoff}{\@dohead@ppfalse}
\newcounter{@ppsavesec}
\newcounter{@ppsaveapp}
\setcounter{@ppsaveapp}{0}
\newcommand{\@ppsavesec}{%
  \if@chapter@pp \setcounter{@ppsavesec}{\value{chapter}} \else
                 \setcounter{@ppsavesec}{\value{section}} \fi}
\newcommand{\@pprestoresec}{%
  \if@chapter@pp \setcounter{chapter}{\value{@ppsavesec}} \else
                 \setcounter{section}{\value{@ppsavesec}} \fi}
\newcommand{\@ppsaveapp}{%
  \if@chapter@pp \setcounter{@ppsaveapp}{\value{chapter}} \else
                 \setcounter{@ppsaveapp}{\value{section}} \fi}
\newcommand{\restoreapp}{%
  \if@chapter@pp \setcounter{chapter}{\value{@ppsaveapp}} \else
                 \setcounter{section}{\value{@ppsaveapp}} \fi}
\providecommand{\appendixname}{APPENDIX}
\newcommand{\appendixtocname}{APPENDICES}
\newcommand{\appendixpagename}{APPENDICES}
\newcommand{\appendixpage}{%
%  \if@chapter@pp \@chap@pppage \else \@sec@pppage \fi
  \if@chapter@pp \else \@sec@pppage \fi
}
\newcommand{\clear@ppage}{%
  \if@openright\cleardoublepage\else\clearpage\fi}

\newcommand{\@chap@pppage}{%
%  \clear@ppage
%  \thispagestyle{plain}%
  \if@twocolumn\onecolumn\@tempswatrue\else\@tempswafalse\fi
%  \null\vfil
%  \markboth{}{}%
  {\centering
%   \interlinepenalty \@M
   \normalfont
   \normalsize \bfseries \appendixpagename\par}%
  \if@dotoc@pp
    \addappheadtotoc
  \fi
  \vfil%\newpage
  \if@twoside
    \if@openright
%      \null
%      \thispagestyle{empty}%
      %\newpage
    \fi
  \fi
  \if@tempswa
    \twocolumn
  \fi
}

\newcommand{\@sec@pppage}{%
  \par
  \addvspace{4ex}%
  \@afterindentfalse
  {\parindent \z@ \raggedright
   \interlinepenalty \@M
   \normalfont
   \normalsize \bfseries \appendixpagename%
   \markboth{}{}\par}%
  \if@dotoc@pp
    \addappheadtotoc
  \fi
  \nobreak
  \vskip 3ex
  \@afterheading
}

\newif\if@pptocpage
  \@pptocpagetrue
\newcommand{\noappendicestocpagenum}{\@pptocpagefalse}
\newcommand{\appendicestocpagenum}{\@pptocpagetrue}
\newcommand{\addappheadtotoc}{%
  \phantomsection
  \if@chapter@pp
    \if@pptocpage
      \addcontentsline{toc}{chapter}{\appendixtocname}%
    \else
      \if@pphyper
        \addtocontents{toc}%
          {\protect\contentsline{chapter}{\appendixtocname}{}{\@currentHref}}%
      \else
        \addtocontents{toc}%
          {\protect\contentsline{chapter}{\appendixtocname}{}}%
      \fi
    \fi
  \else
    \if@pptocpage
      \addcontentsline{toc}{section}{\appendixtocname}%
    \else
      \if@pphyper
        \addtocontents{toc}%
          {\protect\contentsline{section}{\appendixtocname}{}{\@currentHref}}%
      \else
        \addtocontents{toc}%
          {\protect\contentsline{section}{\appendixtocname}{}}%
      \fi
    \fi
  \fi
}

\providecommand{\theH@pps}{\alph{@pps}}

\newcommand{\@resets@pp}{\par
  \@ppsavesec
  \stepcounter{@pps}
  \setcounter{section}{0}%
  \if@chapter@pp
    \setcounter{chapter}{0}%
    \renewcommand\@chapapp{\appendixname}%
     \ifthenelse{\equal{\thechapter}{}}{\renewcommand\thesection{\@Alph\c@section}}{\renewcommand\thechapter{\@Alph\c@chapter}}
  \else
    \setcounter{subsection}{0}%
    \renewcommand\thesection{\@Alph\c@section}%
  \fi
  \if@pphyper
    \if@chapter@pp
      \renewcommand{\theHchapter}{\theH@pps.\Alph{chapter}}%
    \else
      \renewcommand{\theHsection}{\theH@pps.\Alph{section}}%
    \fi
    \def\Hy@chapapp{\appendixname}%
  \fi
  \restoreapp
}

\newenvironment{appendices}{%
  \@resets@pp
  \if@dotoc@pp
    \if@dopage@pp              % both page and toc
      \if@chapter@pp           % chapters
%        \clear@ppage
      \fi
      \appendixpage
    \else                      % toc only
       \if@chapter@pp          % chapters
%         \clear@ppage
       \fi
      \addappheadtotoc
    \fi
  \else
    \if@dopage@pp              % page only
      \appendixpage
    \fi
  \fi
  \if@chapter@pp
    \if@dotitletoc@pp \@redotocentry@pp{chapter} \fi
  \else
    \if@dotitletoc@pp \@redotocentry@pp{section} \fi
    \if@dohead@pp
      \def\sectionmark##1{%
        \if@twoside
          \markboth{\@formatsecmark@pp{##1}}{}
        \else
          \markright{\@formatsecmark@pp{##1}}{}
        \fi}
    \fi
    \if@dotitle@pp
      \def\sectionname{\appendixname}
      \def\@seccntformat##1{\@ifundefined{##1name}{}{\csname ##1name\endcsname\ }%
        \csname the##1\endcsname\quad}
    \fi
  \fi}{%
  \@ppsaveapp\@pprestoresec}

\newcommand{\setthesection}{\thechapter.\Alph{section}}
\newcommand{\setthesubsection}{\thesection.\Alph{subsection}}

\newcommand{\@resets@ppsub}{\par
  \stepcounter{@pps}
  \if@chapter@pp
    \setcounter{section}{0}
    \renewcommand{\thesection}{\setthesection}
  \else
    \setcounter{subsection}{0}
    \renewcommand{\thesubsection}{\setthesubsection}
  \fi
  \if@pphyper
    \if@chapter@pp
      \renewcommand{\theHsection}{\theH@pps.\setthesection}%
    \else
      \renewcommand{\theHsubsection}{\theH@pps.\setthesubsection}%
    \fi
    \def\Hy@chapapp{\appendixname}%
  \fi
}

\newenvironment{subappendices}{%
  \@resets@ppsub
  \if@chapter@pp
    \if@dotitletoc@pp \@redotocentry@pp{section} \fi
    \if@dotitle@pp
      \def\sectionname{\appendixname}
      \def\@seccntformat##1{\@ifundefined{##1name}{}{\csname ##1name\endcsname\ }%
        \csname the##1\endcsname\quad}
    \fi
  \else
    \if@dotitletoc@pp \@redotocentry@pp{subsection} \fi
    \if@dotitle@pp
      \def\subsectionname{\appendixname}
      \def\@seccntformat##1{\@ifundefined{##1name}{}{\csname ##1name\endcsname\ }%
        \csname the##1\endcsname\quad}
    \fi
  \fi}{}

\newcommand{\@formatsecmark@pp}[1]{%
  \MakeUppercase{\appendixname\space
    \ifnum \c@secnumdepth >\z@
      \thesection\quad
    \fi
    #1}}
\newcommand{\@redotocentry@pp}[1]{%
  \let\oldacl@pp=\addcontentsline
  \def\addcontentsline##1##2##3{%
    \def\@pptempa{##1}\def\@pptempb{toc}%
    \ifx\@pptempa\@pptempb
      \def\@pptempa{##2}\def\@pptempb{#1}%
      \ifx\@pptempa\@pptempb
\oldacl@pp{##1}{##2}{\appendixname\space ##3}%
      \else
        \oldacl@pp{##1}{##2}{##3}%
      \fi
    \else
      \oldacl@pp{##1}{##2}{##3}%
    \fi}
}

\endinput
%%
%% End of file `appendix.sty'.
                     %named "appendix.tex"
%\end{appendix}
